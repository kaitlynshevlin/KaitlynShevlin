\documentclass[12pt, letterpaper, titlepage]{article}

\usepackage{amsmath}
\usepackage{booktabs}
\usepackage{amsthm}
\usepackage{graphicx}
\usepackage[margin=1in]{geometry}
\usepackage{hyperref}
\hypersetup{colorlinks = true, linkcolor = blue, citecolor=blue, urlcolor = blue}
\usepackage{natbib}
\usepackage{enumitem}
\usepackage{setspace}

\usepackage[pagewise]{lineno}
\linenumbers*[1]
% %% patches to make lineno work better with amsmath
\newcommand*\patchAmsMathEnvironmentForLineno[1]{%
 \expandafter\let\csname old#1\expandafter\endcsname\csname #1\endcsname
 \expandafter\let\csname oldend#1\expandafter\endcsname\csname end#1\endcsname
 \renewenvironment{#1}%
 {\linenomath\csname old#1\endcsname}%
 {\csname oldend#1\endcsname\endlinenomath}}%
\newcommand*\patchBothAmsMathEnvironmentsForLineno[1]{%
 \patchAmsMathEnvironmentForLineno{#1}%
 \patchAmsMathEnvironmentForLineno{#1*}}%

\AtBeginDocument{%
 \patchBothAmsMathEnvironmentsForLineno{equation}%
 \patchBothAmsMathEnvironmentsForLineno{align}%
 \patchBothAmsMathEnvironmentsForLineno{flalign}%
 \patchBothAmsMathEnvironmentsForLineno{alignat}%
 \patchBothAmsMathEnvironmentsForLineno{gather}%
 \patchBothAmsMathEnvironmentsForLineno{multline}%
}



\title{Survey of Misuses of the KS-Test}

\author{Anthony Zeimbekakis and Jun Yan\\
\href{mailto:anthony.zeimbekakis@uconn.edu}{\nolinkurl{anthony.zeimbekakis@uconn.edu}}\\
Department of Statistics, University of Connecticut}
\date{November 8, 2021}

\begin{document}
\maketitle

\doublespace

\begin{abstract}
The Kolmogorov–Smirnov (KS) test is one the most popular goodness-of-fit tests for comparing a sample with a hypothesized parametric distribution. Nevertheless, it has often been misused. The standard one-sample KS test applies to independent, continuous data with a hypothesized distribution that is completely specified. It is not uncommon, however, to see in the literature that it was applied to dependent, discrete, or rounded data, with hypothesized distributions containing estimated parameters. For example, it has been “discovered” multiple times that the test is too conservative when the parameters are estimated [e.g. \citet{Steinskog}]. This paper aims to survey the misuses of the KS test, demonstrate their consequences through simulation, and provide remedies as needed.
\end{abstract}


\hypertarget{sec:intro}{%
\section{Introduction}\label{sec:intro}}

Introuction here.

\hypertarget{sec:litrev}{%
\section{Problem Under Serial Dependence}\label{sec:litrev}}

Problem Under Serial Dependence here.

\begin{table}[ht]
\centering
\begin{tabular}{rlr}
  \hline
 & Tests\_Samples & Power\_Level \\ 
  \hline
1 & 10000, 10 & 0.95 \\ 
  2 & 10000, 100 & 0.96 \\ 
  3 & 10000, 1000 & 0.95 \\ 
  4 & 10, 10000 & 0.90 \\ 
  5 & 100, 10000 & 0.92 \\ 
  6 & 1000, 10000 & 0.94 \\ 
   \hline
\end{tabular}
\end{table}

\hypertarget{sec:data}{%
\section{Simulation}\label{sec:data}}

Simulation here.

\hypertarget{sec:methods}{%
\section{Real Data Analysis}\label{sec:methods}}

Real Data Analysis here.

\hypertarget{sec:disc}{%
\section{Conclusion}\label{sec:disc}}

Conclusion here.

Here is an example citation: \citet{Steinskog}

\bibliographystyle{chicago}
\bibliography{citations.bib}


\end{document}
